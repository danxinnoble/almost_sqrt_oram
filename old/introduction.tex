\section{Introduction}

Private Information Retrieval (PIR) is a method by which
a client can access an item in a database, without the server
or servers who own the database knowing which item was accessed.

There are two typical frameworks in which this is analyzed.
The first is the multi-server (typically 2-server) model. 
In this case, it is in fact possible to have
a communication protocol that is information-thoeretically 
secure.
For instance Dvir et al. \cite{dvir20162} have a 2-server
information-theoretic PIR protocol with total communication
$n^{O(\sqrt{\log\log(n)/\log(n)})}$.

The second framework is in the single-server model.
In this case it is impossible to acheive information-theoretic
security. However a number of works achieve security against
a computationally-bounded adversary. 
This is sometimes referred to as CPIR (Computational PIR).
There are a number of schemes, many of them reasonably efficient
based on a variety of assumptions.
For instance Lipmaa \cite{lipmaa2005oblivious} presents a one-server PIR 
protocol that requires $\Theta(k\log^2(n) + l \log(n))$ communication
per request (for $n$ items, of length $l$ with security parameter $k$),
requires poly-logarithmic computation by the client and is based
on the Decisional Composite Residuosity assumption.

It is of course also possible to create multi-server PIR protocols
that assume security against computationally-bounded adversaries.
For instance, the protocol of Boyle et al. \cite{boyle2016function}, 
which only requires $O(\log(n))$
communication between the client and the servers,
$O(\log(n))$ client computation and can be based on any
one-way function.

Additionally, if the array is distributed among a number of servers,
we can also have the related primitive of PIR-writes.
If the servers do not have access to the data directly, but rather
the data from servers must be combined in some way to produce
the data, then there can also be protocols for clients to update
items in the database. 
These were introduced by Ostrovsky and Shoup \cite{ostrovsky1997private}
who viewed PIR and PIR-writes as accesses and updates to a replicated
database.

A challenge in PIR is that it is difficult to construct a protocol
in which the amount of \emph{computation} that the server, or servers,
need to do is sub-linear in the size of the array.
Biemel et al. \cite{beimel2004reducing} observed that this is
impossible if they only store the array as-is,
but present a 2-party PIR that acheives sub-linear computation per
query by first performing some some super-linear processing on the array.

A related primitive is ORAM. This allows a client to outsource
their memory to an untrusted server in such a way that their
requests to the server leak nothing about their actual
access patterns of the data (and, of course, leak nothing about
the data). 

Some subtleties should be observed about the differences
between PIR and ORAM.
PIR only allows read-access.
ORAM only allows access to data that has already been written
by the client. 
Also, a PIR may be accessed by multiple clients,
whereas ORAM implementations are often restricted to one client.
Therefore, PIR does not provide ORAM, nor vice versa. 
However, one may be useful in constructing another.
For instance Kushilevitz and Mour constructed a very elegant
and efficient distributed ORAM protocol using PIR.
Their protocol involves four servers, at most one of which is 
malicious, and uses only PIR reads and writes to modify the ORAM
\cite{kushilevitz2018sub}.

In this work we seek to explore the opposite: 
how can ORAM (and MPC) be used to improve PIR protocols?
This requires communication between the servers.
This is an additional cost not experienced by most
PIR protocols, which typically do not require the servers
to even be aware of each other.
However, we believe this extra cost will pay off in other metrics.
