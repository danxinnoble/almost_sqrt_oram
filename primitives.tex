
\section{Primitives}

In this section we describe a number of existing primitives which are
used in our ORAM construction.

\subsection{Oblivious Pseudo-random function}

A Pseudo-Random Function (PRF) is a keyed function such that
any party possessing the key may easily compute the function,
but such that the output of the function will appear random 
to any party who does not possess the key.
An Oblivious PRF is one in which this computation is performed
as a secure multiparty computation.
Specifically, we require a protocol in which the key, input
and output are all secret-shared among a number of parties.

For this application it is in fact sufficient to use the simplest possible
OPRF, namely a look-up table on random values.
While this costs $\sqrt{n}$ secure-comparisons and 
$\sqrt{n}$ muxes per OPRF evaluation,
each ORAM access will only require $3$ such evaluations,
and each ORAM shuffle will only require $3 \sqrt{n}$ such evaluations.
This therefore does not affect the asymptotic performance
of the protocol, and the constants are small.

Of course, other OPRFs may be more efficient and since the design
is modular they can easily be substituted in.

\begin{algorithm}
\caption{OPRF}
\label{alg:oprf}
\begin{algorithmic}[1]

\Procedure{OPRF.init}{size} \Comment{For ORAM, $size = \sqrt{n}$}
\For{$i \gets 1$  to $size$} 
    \State $table[i] \gets^r [size]$
\EndFor
\State \textbf{return} self
\EndProcedure

\State

\Procedure{OPRF.eval}{x}
\State $res \gets 0$
\For{$i \gets 1$ to $size$}
    \OblivIf{$i = x$}
    \State $res \gets table[i]$
    \EndOblivIf
\EndFor
\State \textbf{return} res
\EndProcedure

\end{algorithmic}
\end{algorithm}


\subsection{Feistel Cipher}

The Feistel Cipher allows any PRF $F : [m] \rightarrow[m]$ to be transformed
into a PRP $G : [m] \times [m] \rightarrow [m] \times [m]$. The technique is as follows.

Let $i = (L, R)$, where $i \in [m] \times [m]$ and $L, R \in [m]$.
Let $B_k(i) = (R, L + F_k(R))$.
$B_k$ is clearly invertible and maps $[m] \times [m]$ to $[m] \times [m]$, 
so is a permutation.
However, it is clearly not pseudo-random, since the last element of $i$ 
completely reveals the first element of $B_k(i)$.
Nevertheless, it turns out that if this process is repeated three times,
using three independently chosen keys, the composition is pseudo-random,
i.e. $G_{k_1, k_2, k_3} \defined B_{k_3} \circ B_{k_2} \circ B_{k_1}$ is a PRP
\cite{luby1988construct}.

This is described as an algorithm in the following figure,
where the PRF is treated in a black-box manner.

\begin{algorithm}
\caption{OPRP}
\label{alg:oprp}
\begin{algorithmic}[1]

\Procedure{OPRP.init}{n, $F_1$, $F_2$, $F_3$}
\State \textbf{return} self   \Comment{Remember initialization variables}
\EndProcedure

\Procedure{OPRP.eval}{x}
\State $L_0 \gets x \bmod \sqrt{n}$
\State $R_0 \gets \lfloor \frac{x}{\sqrt{n}} \rfloor$

\For{$i \gets 1$ to $3$}
\State $L_i \gets R_{i-1}$
\State $R_i \gets L_{i-1} + F_{i}.eval(R_{i-1}) \bmod \sqrt{n}$
\EndFor

\State $res \gets L_3\sqrt{n} + R_3$
\State \textbf{return} $res$
\EndProcedure

\end{algorithmic}
\end{algorithm}
\subsection{Oblivious Sorting}

Oblivious sorting is fundamental to many ORAM protocols and is often
a bottle-neck in the secure computation.
However, we can utilize the fact that every sort we do will be sorting
based on an array that contains every possible element exactly once.
Therefore, we can randomly permute the elements first,
then reveal the field to be sorted on and then sort in the clear.

The function takes at least two arguments. 
The first, $A$, is an array of the indices to be sorted on.
Their distribution (but not order) is public and there are no duplicates.
The remaining arguments are arrays of the same length as the index array,
which should be sorted according to the index array.
The protocol is presented in Figure~\ref{fig:OblivSort}.


\begin{algorithm}
\caption{oblivSort: Sort based on indexes with a public distribution}
\label{fig:OblivSort}
\begin{algorithmic}[0]

\Procedure{oblivSort}{A, B}
\State $(\hat{A} , \hat{B}) = shuffle(A, B)$
\State $\tilde{A} = reveal(\hat{A})$
\State $\bar{B} = sort(\tilde{A}, \hat{B})$
\State \textbf{return} $\bar{B}$
\EndProcedure

\end{algorithmic}
\end{algorithm}

\subsection{Random Shuffle}

There are many protocols for a random shuffle.
We will use a Waksman permutation network.
\footnote{Waksman permutation networks are very good for shuffling an array,
however evaluating the permutation on a particular point cannot be done
efficiently. Hence, we do not use Waksman networks to perform the ORAM permutation.}
This is a $O(n \log{n})$-sized network of switches that allows computationation
of an arbitrary permutation by setting the switch-bits appropriately.
An important observation is that setting the switch-bits randomly
does not result in a random permutation.
Therefore, for the MPC setting where at most $t$ players are dishonest,
$(t+1)$ players should each choose switch-bits and have a Waksman permutation
evaluated on the array, in turn, using these switch bits.  
In an arithmetic ciruit MPC framework, this can be done using $O( (t+1) n \log{n})$ multiplications.

\subsection{Linear ORAM}

This is the trivial ORAM solution, in which the element is searched for
in every location.
This is used for inserting elements into the stash and finding
elements that were inserted.
Of course, a more efficient ORAM could be used to implement the stash,
for instance the square-root ORAM could be used recursively.

For the stash application, this may be slightly optimized compared to the 
general case since the type of access (read vs write) can be public.

\begin{algorithm}
\caption{LinearORAM}
\label{alg:obliv_set}
\begin{algorithmic}[1]

\Procedure{LinearORAM.init}{$max\_size$}  

\State $stash \gets [ \bot $ for $0 \leq i < max\_size]$
\State $nElem \gets 0$
\State \textbf{return} self
\EndProcedure

\State

\Procedure{LinearORAM.add}{$(\secret{x}, \secret{y})$} 
\Comment{Assumes fewer than $max\_size$ elements have already been added}
\State $stash[nElem + 1] \gets (\secret{x}, \secret{y})$
\State $nElem \gets nElem + 1$
\EndProcedure

\State

\Procedure{LinearORAM.empty}{}
\State $stash \gets [ \bot $ for $0 \leq i < nElem]$
\State $nElem \gets 0$
\EndProcedure

\State

\Procedure{LinearORAM.get}{$\secret{key}$}
\State $\secret{res} \gets \bot$
\For{$j \gets 1$ to $nElem$}
    \State $(\secret{x}, \secret{y}) \gets stash[j]$
    \OblivIf{$\secret{key} = \secret{x}$}
        \State $\secret{res} \gets \secret{y}$
    \EndOblivIf
\EndFor
\State \textbf{return} res
\EndProcedure

\end{algorithmic}
\end{algorithm}


