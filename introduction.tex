\section{Efficient, simple, Square-Root ORAM}

The original ORAM paper \cite{ostrovsky1997private} presented a protocol for performing
oblivious-RAM with amortized cost $O(\sqrt{n} \log{n})$.

In the secure-computation setting, a direct application of this
protocol would require many evaluations of an oblivious PRF (OPRF).
While in principle this is a constant-cost operation, in practice evaluating an OPRF is expensive. For instance, in the sem-honest, 2-party setting, the communication cost of oblivious versions of standard PRFs is around 170 Kb and around 20 Kb for MPC-optimized PRFs (\cite{hemenway2019private} Table 1) % This should be updated to reference the actual papers and use more recent results.

A different approach to square-root ORAM was therefore explored by Zahur et al is to use a recursive solution to store a look-up table for the evaluations of the PRF.

We adopt a different approach. We use OPRFs as in the original protocol.
However we take advantage of a number of facts to make the OPRFs more efficient.
Firstly, each PRF will only be evaluated on $\sqrt{n}$ elements.
Therefore, the PRF does not need to be secure up to an arbitrary number of 
evaluations. In deed, it can be rendered completely insecure if evaluated
on $\sqrt(n) + 1$ points and still be secure for our purposes.
Secondly the PRF is evaluated over a small input.
Thirdly, if the PRF is, in fact, a permutation and 
if the function is evaluated for \emph{every} point in the range,
the distribution of the output (when sorted or shuffled again)
does not leak any information about the function.
We achieve this using Feistel's technique to transform a PRF into a PRP.

