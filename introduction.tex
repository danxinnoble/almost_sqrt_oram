Private Information Retrieval (PIR) is a method by which
a client can access an item in a database, without the server
or servers who own the database know which item was accessed.

There are two typical frameworks in which this is analyzed.
The first is the multi-server (typically 2-server) model. 
In this case, it is in fact possible to have
a communication protocol that is information-thoeretically 
secure, such as the work of Boyle et al. \cite{boyle2016function}, 
which only requires $O(\log(n))$
communication between the client and the servers.
There 

The second framework is in the single-server model.

A related primitive is ORAM. This allows a client to outsource
their memory to an untrusted server in such a way that their
requests to the server leak nothing about their actual
access patterns of the data (and, of course, leak nothing about
the data). 

Some subtleties should be observed about the differences
between PIR and ORAM.
PIR only allows read-access.
ORAM only allows access to data that has already been written
by the client.  
