\section{Introduction}

Oblivious RAM (ORAM) is a technique by which
a client can access a virtual memory of size $m$,
by accessing some physical memory 
in such a way that the physical memory accesses reveal
nothing about the virtual memory accesses.
This is often understood in a setting in which a client
outsources encrypted data to a server.
While encrypting the data hides the content of the data,
the client's access patterns may leak sensitive information.
Another setting in which ORAM is especially useful
is in secure multi-party computation (MPC).
These are protocols that allow a set of parties to compute
a function on sensitive data without revealing the data itself.
Sometimes the sensitive data may be owned by a particular
party, but more generally, it may be secret-shared between
the parties such that no single party can access any sensitive data.
While efficient protocols exist for computing basic operations,
a significant challenge arises if the indexes of memory accesses
need to be kept secret.
A potential solution to this is to implement an ORAM protocol
inside of the MPC, so that the parties' memory accesses
do not leak any information about sensitive data.

Multiple ORAM schemes have been proposed through the years,
generally falling either into the category of 
tree-based designs \cite{shi2011oblivious, stefanov2013path, wang2015circuit},
where newly-written elements are placed at the top of a tree
and blocks are pushed downwards according to pre-assigned random values,
or heirarchical designs \cite{ostrovsky1997private, asharov2018optorama}
where newly-written elements are stored in a stash
which is, when full, reshuffled into larger stashes
of exponentially-increasing size. 
These have gone from costing $\mathcal{O}(\log(m)^3)$ communication
per access to costing $\mathcal{O}(\log(m))$ 
communication per access.

However, it was observed by Zahur et al. \cite{zahur2016revisiting}
that since the constants in these protocols are large,
it is in practice more efficient, for certain ranges of $m$,
to use protocols that are definitely sub-optimal asymptotically.

For instance, the original ORAM paper \cite{ostrovsky1997private} 
presented a protocol with amortized cost $\mathcal{O}(\sqrt{m} \log{m})$
per access.
Zahur et al. present a modification of this protocol
that out-performs the then-optimal Circuit ORAM \cite{wang2015circuit}
for certain useful ranges of $m$.
Zahur et al. observed that evaluating multiple 
oblivious pseudo-random functions (OPRFs)
during the shuffle step was a significant bottleneck in the original 
square-root ORAM scheme.
While, in theory each OPRF evaluation is constant cost, the constant is high.
Zahur et al. managed to avoid this by using a recursive table look-up.

We follow in the vein of Zahur et al. but try to make the OPRFs 
efficient in a different way.
We replace the OPRFs with Oblivious Pseudo-Random Permutations (OPRPs)
and evaluate these using a Feistel Cipher, which in turn
uses OPRFs, but does not need to evaluate them on as many 
inputs as the original square-root ORAM protocol.

Unfortunately, the security of the Feistel Cipher OPRP requires shuffling to occur
once every $o(\sqrt{m})$ accesses,
leading to a scheme that has complexity $\omega(m^{\frac{1}{2}})$.
Additionally, the appropriate parameters to ensure concrete security
for Feistel Ciphers are not well understood.
Improving the understanding of the security of this type of construction
as well as methods for avoiding the $\mathcal{O}(\sqrt{m})$ bound
remain as future work.

We implemented the MPC-ORAM protocol 
(with parameters for which security could not be proven)
in SCALE-MAMBA and found that, for roughly $m > 4000$ this 
scheme was more efficient than the na\"{i}ve Linear ORAM scheme.
