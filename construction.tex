
\section{Construction}

We now describe our construction of an oblivious RAM that uses PRFs
but has a much smaller cost during the shuffle step.

\begin{algorithm}
\caption{SqrtORAM}
\label{alg:oram}
\begin{algorithmic}[1]

\State ROUNDS=7
\State EPSILON= $\frac{1}{8}$
\State c=1

\State

\Procedure{SqrtOram.get\_max\_evals}{n}
\State q = $\lfloor c n^{\frac{1}{2} - EPSILON} \rfloor$
\State \textbf{return} q
\EndProcedure

\State

\Procedure{SqrtOram.init}{m, data\_orig}
\Comment{$m$ is the size of the array}

\State $n \gets$ smallest $n= 2^{2k}$, such that $n - $ this.get\_max\_evals(n) $ < m$ and $k \in \mathbb{N}$.
\State $q \gets$ this.get\_max\_evals(n)

\For{ $i \gets 1$ to $m$}
   \State $data[i] = data\_orig[i]$
\EndFor

\For{ $i \gets m+1$ to $n$}
   \State $data[i] = \bot$
\EndFor
   
\For{ $i \gets 1$ to $n$}
    \State $ind[i]  = (\lfloor{ \frac{i}{\sqrt{n}}} \rfloor ,  i \bmod \sqrt{n}) $
\EndFor

\State this.shuffle()

\State \textbf{return} self
\EndProcedure

\State

\Procedure{SqrtOram.shuffle}{}

\State $(ind_0, data_0) = OblivSort(ind, (ind, data))$ \label{line:first_sort}

\State $F = [$ OPRF.init() for $i \gets 1$ to $ROUNDS]$

\For{ $k \gets 1$ to $ROUNDS$}
    \For{ $j \gets 1$ to $\sqrt{n}$}
        \State $A_k[j] = F_k.eval(j)$ \label{line:prf_eval}
        \For{$i \gets 1$ to $\sqrt{n}$}
            \State $B_k[\sqrt{n} i + j] = (j, i + A_k[j] \bmod \sqrt{n})$
        \EndFor
    \EndFor
 
    \State $(ind_k, data_k) = OblivSort(B_k, (ind_{k-1}, data_{k-1}))$
        \label{line:second_sort}
  
\EndFor

\State $stash \gets LinearORAM.init(\sqrt{n})$

\State $oprp \gets OPRP.init(n, F)$

\State $count = 0$

\State \textbf{return} self

\EndProcedure

\State

\algstore{myalg}

\end{algorithmic}
\end{algorithm}

\begin{algorithm}
\begin{algorithmic}[1]
\algrestore{myalg}

\Procedure{SqrtOram.Access}{i}

\If{$count = $ this.get\_max\_evals(n)}
    \State this.shuffle()
\EndIf

\State $res \gets stash.get(i)$

\If{ $res \neq \bot$}
    \State $p \gets oprp.eval(m + count)$
\Else
    \State $p \gets oprp.eval(i)$
\EndIf

\State $datum \gets data[p]$
\State $stash.add((i, datum))$

\If{ $res \neq \bot$ }
    \State $res \gets res$
\Else
    \State $res \gets datum$
\EndIf

\State $count \gets count + 1$ 

\State \textbf{return} res
\EndProcedure

\end{algorithmic}
\end{algorithm}


The construction follows the original square-root ORAM construction
\cite{ostrovsky1997private} in general. 
However it has a much more efficient shuffling step.

The original ORAM construction stores each item $i$
in a secret location, $OPRF_k(i)$, in a table where
the key, $k$, for the PRF is secret and the PRF is evaluated obliviously.
In order to avoid leakage from accessing the same element twice,
accessed elements are stored in a stash. 
In each access, the item is first searched for in the stash.
If it found there, then a dummy item is accessed in the table.

In order to efficiently reshuffle the table, we use a Feistel-constructed
PRP as the PRF.

The critical observation is that if $B_k : [n] \rightarrow [n]$ is a permutation, 
then the set $\{B_k(j) \}_{j=1 \ldots n}$ is exactly the set $\{1 \ldots n\}$.
Therefore, revealing the contents (though not the order) of
$\{B_k(j)\}_{j=1 \ldots n}$ reveals nothing.

We use this fact in the OblivSort function in Algorithm \ref{fig:OblivSort}.
The data is first shuffled using several Waksman permutation networks
(one from each player).
Then the evaluations of $B_k$ are revealed and sorting is performed in the clear
(carrying the corresponding payloads).

To perform the psuedo-random permutation on every element, 
$ROUNDS$ permutations, $B_1 \ldots B_{ROUNDS}$ are applied in turn.
The payload is always kept secret, 
but in each iteration the indexes are shuffled, revealed and sorted.
To apply $B_k$, the corresponding OPRF, $F_k$ is applied to every possible string
in $[\sqrt{n}]$. 
Then $(L_k, R_k) = (R_{k-1}, L_{k-1} + F_k(R_{k-1}) \bmod \sqrt{n})$ can 
be evaluated for every $(L_{k-1}, R_{k-1}) \in [\sqrt{n}] \times [\sqrt{n}]$.
Observe that the OPRF $F_k$ need only be evaluated on each value of
$R_{k-1}$ once, so need only be evaluated $\sqrt{n}$ times.
Therefore to perform $ROUNDS$ permutations, a total of $ROUNDS \sqrt{n}$ OPRF
evaluations are needed.



