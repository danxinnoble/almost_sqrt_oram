\section{Future Work}

This work seeked to implement square-root ORAM using an 
efficient, secure OPRP on a small domain.

However, the OPRP that was chosen, a Feistel cipher with 7 rounds,
did not provide concrete security guarantees,
and only provided asymptotic security guarantees when the number
of queries was  $o(\sqrt{n})$, which meant that
the asymptotic cost per access needed to be $\omega(\sqrt{n})$.

As such, finding an efficient PRP with concrete security for $\mathcal{O}(\sqrt{n})$
evaluations would be of great use.
A number of possible future directions could be explored to address this problem.

\begin{itemize}

\item
Unbalanced Feistel networks may have better security than balanced ones.
For instance, the bound of $\mathcal{O}(\sqrt{n})$ queries can be broken
in the case of a fully unbalanced Feistel cipher, where instead
of half of the bits in each round contributing to the PRF, 
all but one of the bits do \cite{morris2009encipher}.
Unfortunately, the results of this paper require non-constant number of rounds, 
namely $\mathcal{O}(\log{n})$.

\item
There has been some work into sampling from $k$-wise almost independent permutations.
For instance, Kaplan et al. \cite{kaplan2009derandomized} show how to 
\emph{represent} a random permutation from a set of $k$-wise 
$\delta$-close-to-indendent
permutaions on $n$ using memory $\mathcal{O}(k\log{(n)} + \log(\frac{1}{\delta}))$.
It is not clear, though, how efficiently such a permuation could be evaluated
on a single element, or how efficiently an array could be sorted on such a 
permutation.
This result was for obtained for space-bounded adversaries,
it is again not clear whether the scheme could be more efficient
against a computationally-bounded adversary.

\item 
Certain tailor-made protocols for evaluating specific functions
inside of an MPC may be much more efficient than performing a general MPC
computation.
For instance, a shuffle in the (3, 1) security setting can be 
efficiently performed by secret-sharing the data between each pair
of parties, in turn, and each pair of parties performing the shuffle.
However, it is not clear how to evaluate the permutation on any
specific point without doing an oblivious computation.

\item
Another avenue that could be explored is causing the client
to query multiple indexes per access:
the index that they wish to access plus a random sequence of other indexes.
An adversary would not know which of the accesses was the real one.
This may significantly reduce the adversary's ability 
to distinguish the permutation from a random permutation,
allowing the security to be increased, or the PRP to be made more efficient.

\end{itemize}

Additionally, a number of simple modifications could be taken to make
the Feistel-cihper scheme more efficient.

\begin{itemize}

\item
The PRF was instantiated using a look-up table for a truly random function.
This is overkill for security and is likely not the most efficient solution
(though it does have the benefit that the PRF can be evaluated on 
public values during the Feistel cipher evaluation).

The scheme could take advantage of the fact that SCALE-MAMBA supports
evaluation of arbitrary circuits, and includes circuits for 
standard PRFs such as AES. 
Alternatively, the design could make use of a PRP that is specifically
designed for operations in a prime field (e.g. \cite{albrecht2016mimc})
which are natively supported in SCALE-MAMBA.
\footnote{
A PRF that applies to a large space such as $2^{128}$
can be used for the PRFs in the Feistel cipher, simply by truncating the bits.
However, a PRP over such a large space cannot be used to 
construct our needed PRP over a small space, since
truncating the bits would cause it to no longer be a permutation.}

\item
In the setting of 3 parties, one of whom is malicious,
a more efficient shuffling scheme can be used in the shuffle-then-sort protocol.
Namely, the data can be secret shared between a pair of parties,
who can locally shuffle the array according to a secret they decide together.
If each pair of parties does this, an unknown random shuffle is performed.
This requires 3 resharings of $n$ elements, rather than 3$\frac{n}{2} (2\log(n) - 1)$ evaluations of gates in a Benes network.

\item
ORAM schemes typically make use of the data being stored in ``blocks''.
In this scheme, only chunks of data of size 32 bytes were stored.
If data were stored in larger blocks, the number of blocks
that would need to be stored would be reduced.


\end{itemize}



